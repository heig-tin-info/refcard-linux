\documentclass[10pt]{article}

\usepackage{graphicx}
\usepackage{tabularx}

\usepackage[a4paper,landscape,margin=10mm]{geometry}
\usepackage{multicol}
\setlength{\columnsep}{5mm}
\setlength{\columnseprule}{0.4pt}

\usepackage{fontspec}
% \setmainfont{TeX Gyre Heros}
% \setsansfont{TeX Gyre Heros}
% \setmonofont{Inconsolata}
\renewcommand{\familydefault}{\sfdefault}

\usepackage[french]{babel}

\usepackage{microtype}
\usepackage{enumitem}
\setlist{nosep,leftmargin=1.1em}
\setlength{\parindent}{0pt}
\setlength{\parskip}{2pt}

\usepackage{listings}
\lstset{
  basicstyle=\ttfamily\footnotesize,
  columns=flexible,
  keepspaces=true,
  upquote=true,
  breaklines=true
}

\usepackage[os=win]{menukeys}
\renewmenumacro{\keys}[+]{roundedkeys}

\renewcommand{\familydefault}{\sfdefault}

\usepackage{titlesec}

% Tailles des titres (s'applique aussi aux versions étoilées *)
\titleformat{\section}
  {\sffamily\bfseries\fontsize{12}{15}\selectfont} % ajoute  si tu veux en gras
  {}{0pt}{}

\titleformat{\subsection}
  {\sffamily\bfseries\fontsize{10}{12}\selectfont}
  {}{0pt}{}

\titleformat{\subsubsection}
  {\sffamily\bfseries\fontsize{8}{10}\selectfont}
  {}{0pt}{}

% Espacements (les tiens, inchangés)
\titlespacing*{\section}{0pt}{0.5ex plus .1ex minus 0.1ex}{0.5ex plus .1ex minus 0.1ex}
\titlespacing*{\subsection}{0pt}{0.5ex plus .1ex minus 0.1ex}{0.5ex plus .1ex minus 0.1ex}
\titlespacing*{\subsubsection}{0pt}{0.5ex plus .1ex minus 0.1ex}{0.5ex plus .1ex minus 0.1ex}


\pagenumbering{gobble}

% \titlespacing*{\section}{0pt}{0.5ex plus .1ex minus 0.1ex}{0.5ex plus .1ex minus 0.1ex}
% \titlespacing*{\subsection}{0pt}{0.5ex plus .1ex minus 0.1ex}{0.5ex plus .1ex minus 0.1ex}
% \titlespacing*{\subsubsection}{0pt}{0.5ex plus .1ex minus 0.1ex}{0.5ex plus .1ex minus 0.1ex}

\newlength\mybaselinestretch
\mybaselinestretch=0pt plus 0.02pt\relax
\addtolength{\baselineskip}{\mybaselinestretch}

\setlength\parindent{0pt}
\setlength\tabcolsep{1.5pt}
\setlength{\columnseprule}{0.4pt}

\def\revision{1.0}
\def\revisiondate{2024-06-12}

\begin{document}

\begin{multicols}{4}

\begin{tabularx}{\columnwidth}{lX}
    \raisebox{-\totalheight}{\includegraphics[width=1cm]{assets/heig-vd-black.pdf}} &
    \begin{center}
      {\Large \bf Linux / Unix} \\
      HEIG-VD -- version \revision \ -- \revisiondate \\
    \end{center}
\end{tabularx}

\section*{Raccourcis clavier Bash}
\subsection*{Mouvements}
\keys{\ctrl + B} : recule d'un caractère \\
\keys{\ctrl + F} : avance d'un caractère \\
\keys{\ctrl + A} : début de ligne \quad
\keys{\ctrl + E} : fin de ligne \\
\keys{\Alt}{}+\keys{B} : recule d'un mot \quad
\keys{\Alt}+\keys{F} : avance d'un mot \\[2pt]

\subsection*{Édition}

\keys{\ctrl + H} : suppr. caractère avant (\keys{\backspace}) \\
\keys{\ctrl + D} : suppr. caractère sous le curseur (\keys{\del}) \\
\keys{\ctrl + U} : coupe du début au curseur \\
\keys{\ctrl + K} : coupe du curseur à la fin \\
\keys{\ctrl + W} : coupe le mot avant le curseur (\keys{\Alt}+\keys{\backspace}) \\
\keys{\ctrl + Y} : colle le dernier texte coupé \\
\keys{\ctrl + T} : échange le caractère courant avec le précédent \\[2pt]

\textbf{Exécution / complétion :}\\
\keys{\ctrl + J}, \keys{\ctrl + M} : exécuter (\keys{\return}) \\
\keys{\ctrl + I} : complétion (équiv. \keys{\tab}) \\
\keys{\tab}{} : auto-complétion commandes/fichiers \\[2pt]

\textbf{Marqueur :}\\
\keys{\ctrl + @} : place un marqueur \\
\keys{\ctrl + X},\keys{\ctrl + X} : échange curseur \(\leftrightarrow\) marqueur \\[2pt]

\textbf{Contrôle :}\\
\keys{\ctrl + C} : interrompre \\
\keys{\ctrl + Z} : suspendre \\
\keys{\ctrl + L} : effacer l'écran (\lstinline|clear|) \\
\keys{\ctrl + D} : envoyer EOF (fermer le shell si vide) \\[2pt]

\textbf{Historique / recherche :}\\
\keys{\ctrl + R} : recherche historique \\
\keys{\ctrl + G} : annuler la recherche \\
\keys{\ctrl + P} : commande précédente \\
\keys{\ctrl + N} : commande suivante \\
\keys{\ctrl + S} : \emph{flux} pause (débloquer avec \keys{\ctrl + Q}) \\
\keys{\Alt}+\keys{<} : début de l'historique \\
\keys{\Alt}+\keys{>} : fin de l'historique \\
\keys{\Alt}+\keys{P} : préc. dans l'historique \quad
\keys{\Alt}+\keys{N} : suiv. dans l'historique \\[2pt]

\textbf{Casse / mots :}\\
\keys{\Alt}+\keys{U} : majuscules (mot après) \\
\keys{\Alt}+\keys{L} : minuscules (mot après) \\
\keys{\Alt}+\keys{C} : capitalise la première lettre \\
\keys{\Alt}+\keys{D} : supprime le mot après \\
\keys{\Alt}+\keys{.} : insère le dernier argument \\

\section*{Placeholders courants}
\lstinline|~| : $HOME \quad$
\lstinline|.| : répertoire courant \\
\lstinline|..| : répertoire parent \\
\lstinline|!!| : dernière commande \\
\lstinline|!n| : commande \#n de l'historique \\
\lstinline|$?| : code retour dernière commande \\
\lstinline|$0| : nom du script \\
\lstinline|!*| : tous les arguments de la commande précédente \\
\lstinline|^abc^123| : remplace \lstinline|abc| par \lstinline|123| dans la dernière commande \\
\lstinline|`cmd`| (obsolète) \(\rightarrow\) préférer \lstinline|$(cmd)| \\
\lstinline|$(cmd)| : substitution de commande \\
\lstinline|${var}| : variable \\
\lstinline|{a,b,c}| : expansion par alternatives \\
\lstinline|{1..5}|, \lstinline|{a..e}| : expansions de séquence \\
\lstinline|$(cmd1; cmd2)| : sous-coquille \\
\lstinline|$(cmd &)| : arrière-plan (sous-coquille) \\
\lstinline|$(( expr ))| : arithmétique \\

\section*{Redirections / pipes}
\lstinline|>| : stdout \textrightarrow{} fichier (écrase) \\
\lstinline|>>| : stdout \textrightarrow{} fichier (ajoute) \\
\lstinline|2>| : stderr \textrightarrow{} fichier (écrase) \\
\lstinline|&>| : stdout+stderr \textrightarrow{} fichier (écrase) \\
\lstinline|&>>| : stdout+stderr \textrightarrow{} fichier (ajoute) \\
\lstinline|<| : stdin depuis fichier \\
\lstinline|<<| : heredoc \quad \lstinline|<<<| : here-string \\
\lstinline!|! : pipe \quad \lstinline!|&! : pipe stdout+stderr \\
\lstinline|2>&1| : stderr \textrightarrow{} stdout \\
\lstinline|cmd1 | cmd2 | cmd3| \\

\section*{Commandes de base}
\lstinline|uname -a| \quad
\lstinline|date| \quad
\lstinline|cal| \quad
\lstinline|whoami| \\

\section*{Fichiers / répertoires}
\lstinline|pwd| \\
\lstinline|ls| (\lstinline|-l -a -h -R -t -r -Q -i -d -S -1|) \\
\lstinline|cd [répertoire]| \\
\lstinline|mkdir [nom]| (\lstinline|-p|, \lstinline|-m [perms]|) \\
\lstinline|rmdir [nom]| \\
\lstinline|rm [fichier]| (\lstinline|-i -f -v -r|) \\
\lstinline|cp [src] [dst]| (\lstinline|-i -f -v -r|) \\
\lstinline|mv [src] [dst]| \\

\section*{Exécution de commandes}
\lstinline|source [fichier]| \\
\lstinline|./[fichier]| (exécutable) \\

\section*{Consultation de fichiers}
\lstinline|cat| \quad
\lstinline|less| \quad
\lstinline|head| \quad
\lstinline|tail| (\lstinline|-n N|, \lstinline|-f|) \\
\lstinline|nano| \quad \lstinline|vi| \quad \lstinline|hexdump -C| \\
\lstinline|file| \quad \lstinline|stat| \quad \lstinline|wc| \\
\lstinline|diff [f1] [f2]| \\

\section*{Flux / petits utilitaires}
\lstinline|touch [f]| \\
\lstinline|echo [-n|-e] "texte"| \\
\lstinline|cat > [f]| (\keys{\ctrl}+\keys{D} pour terminer) \\
\lstinline|cat >> [f]| (\keys{\ctrl}+\keys{D}) \\

\section*{Variables d'environnement}
\lstinline|export VAR=valeur| \quad \lstinline|echo $VAR| \\
\lstinline|env| \quad \lstinline|printenv| \quad \lstinline|unset VAR| \\
\lstinline|$PATH|, \lstinline|$HOME|, \lstinline|$SHELL|, \lstinline|$USER|, \lstinline|$PS1|, \lstinline|$PWD|, \lstinline|$OLDPWD| \\

\section*{Processus / jobs}
\lstinline|ps| \quad \lstinline|top| \quad \lstinline|htop| \\
\lstinline|kill [PID]| \quad \lstinline|killall [nom]| \\
\lstinline|bg| \quad \lstinline|fg| \quad \lstinline|jobs| \\

\section*{Archives / compression}
\textbf{\lstinline|.tar.gz|} : \lstinline|tar -czvf archive.tar.gz [f]| \\
\lstinline|tar -xzvf archive.tar.gz| \\
\textbf{\lstinline|.zip|} : \lstinline|zip archive.zip [f]| \quad \lstinline|unzip archive.zip| \\
\textbf{\lstinline|.tar.xz|} : \lstinline|tar -cJvf| / \lstinline|tar -xJvf| \\
\textbf{\lstinline|.7z|} : \lstinline|7z a archive.7z [f]| \quad \lstinline|7z x archive.7z| \\
\textbf{bsdtar} : \lstinline|bsdtar -xf [archive]|, \lstinline|bsdtar -cf [arch] [f]|, \lstinline|bsdtar -tf [arch]| \\

\section*{Permissions / utilisateurs}
\lstinline|chmod [opts] [perms] [f]| \\
\lstinline|chown utilisateur:groupe [f]| \\
\lstinline|chgrp groupe [f]| \\
\lstinline|umask| \\
\lstinline|sudo [cmd]| \quad \lstinline|su [utilisateur]| \\
\lstinline|who| \quad \lstinline|w| \quad \lstinline|id| \quad \lstinline|groups| \quad \lstinline|passwd| \\
\textit{bits :} 1 x, 2 w, 4 r \\
Exemples : \lstinline|chmod 755 f| (\lstinline|rwxr-xr-x|), \lstinline|chmod 644 f|, \lstinline|chmod u+x f|, \lstinline|chmod g-w f|, \lstinline|chmod o=r f|, \lstinline|chmod a+x f| \\

\section*{less / more / man}
Espace: page suivante \quad \keys{b}: page précédente \\
\keys{\return}{}: ligne suivante \quad \keys{y}: ligne précédente \\
\keys{/}: rechercher \quad \keys{n}/\keys{N}: suivant/précédent \\
\keys{g}/\keys{G}: début/fin \quad \keys{q}: quitter \\
\keys{h}: aide \quad \keys{m}: marque-page \\
\keys{'}: aller au marque-page \quad \keys{`}: revenir \\
\keys{+} \& \keys{-} + numéro : aller à la ligne \\

\section*{Git}
\lstinline|git status| \quad \lstinline|git add [f]| \\
\lstinline|git commit -m "msg"| \\
\lstinline|git push| \quad \lstinline|git pull| \\
\lstinline|git clone [url]| \\
\lstinline|git branch| \quad \lstinline|git checkout [br]| \\
\lstinline|git merge [br]| \\

\section*{Recherche}
\lstinline|grep "texte" [f]| (\lstinline|-P -r -i -v -n -c -l -w -A n -B n -C n -o|) \\
\lstinline|rg "texte" [rep]| (\lstinline|--json|) \\

\section*{Remplacement}
\lstinline|sed 's/ancien/nouveau/g' [f]| (\lstinline|-i -n -e -r|) \\
\lstinline|perl -pe 's/ancien/nouveau/g' [f]| (\lstinline|-i -n -e -l|) \\
\lstinline|perl -0777 -pe 's/ancien/nouveau/gs' [f]| \\

\section*{Utilitaires}
\lstinline|uniq [f]| (\lstinline|-c -d -u -i|) \\
\lstinline|sort [f]| (\lstinline|-r -n -k n -u -t x|) \\
\lstinline|cut -d x -f n [f]| \\
\lstinline|awk '{print $n}' [f]| \\
\lstinline|xargs [cmd]| (\lstinline|-n N -I {}|) \\
\lstinline|tr 'set1' 'set2'| (\lstinline|-d -s|) \\
\lstinline|tee [f]| (\lstinline|-a -i|) \\

\section*{Aide}
\lstinline|man [cmd]| (\lstinline|-k terme|, \lstinline|-f cmd|) \\
\lstinline|apropos [terme]| \quad \lstinline|whatis [cmd]| \\
\textit{Sections :} 1 utilisateur, 2 syscalls, 3 lib, 4 spéciaux, 5 formats, 6 jeux, 7 divers, 8 admin \\

\section*{Réseau}
\lstinline|ping [hôte]| (\lstinline|-c N -i s -s n -t TTL -W s|) \\
\lstinline|ss -tuln| (\lstinline|-t -u -l -n -p|) \\
\lstinline|ip addr| \quad \lstinline|ip route| \\

\section*{Internet}
\lstinline|wget [url]| (\lstinline|-O f -c -q --limit-rate=100k|) \\
\lstinline|curl [url]| (\lstinline|-o f -O -L -I -d data -H 'Hdr: v'|) \\

\section*{Env. virtuels Python}
\lstinline|uv venv venv| \quad \lstinline|source venv/bin/activate| \\
\lstinline|deactivate| \\
\lstinline|pip install pkg| \quad \lstinline|pip uninstall pkg| \\
\lstinline|pip list| \quad \lstinline|uv add pkg| \\
\lstinline|pipx install pkg| \\

\section*{APT (Debian/Ubuntu)}
\lstinline|apt update| \quad \lstinline|apt upgrade| \\
\lstinline|apt install pkg| \quad \lstinline|apt remove pkg| \\
\lstinline|apt search terme| \quad \lstinline|apt show pkg| \\
\lstinline|apt autoremove| (\lstinline|-y --purge|) \\

\section*{Fun}
\lstinline|cowsay "texte"| \\
\lstinline|fortune | cowsay| \\
\lstinline|sl| \quad \lstinline|lolcat| \\

\section*{À installer (exemples)}
\textbf{Manuels FR} : \lstinline|sudo apt install manpages-fr manpages-fr-extra| \\
\textbf{Dev} : \lstinline|sudo apt install build-essential git curl wget vim htop| \\
\textbf{Réseau} : \lstinline|sudo apt install net-tools iputils-ping traceroute dnsutils| \\

\end{multicols}
\end{document}
